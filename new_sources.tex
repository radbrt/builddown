\documentclass[11pt,]{article}
\usepackage[left=1in,top=1in,right=1in,bottom=1in]{geometry}
\newcommand*{\authorfont}{\fontfamily{phv}\selectfont}
\usepackage[]{mathpazo}


  \usepackage[T1]{fontenc}
  \usepackage[utf8]{inputenc}




\usepackage{abstract}
\renewcommand{\abstractname}{}    % clear the title
\renewcommand{\absnamepos}{empty} % originally center

\renewenvironment{abstract}
 {{%
    \setlength{\leftmargin}{0mm}
    \setlength{\rightmargin}{\leftmargin}%
  }%
  \relax}
 {\endlist}

\makeatletter
\def\@maketitle{%
  \newpage
%  \null
%  \vskip 2em%
%  \begin{center}%
  \let \footnote \thanks
    {\fontsize{18}{20}\selectfont\raggedright  \setlength{\parindent}{0pt} \@title \par}%
}
%\fi
\makeatother




\setcounter{secnumdepth}{0}

\usepackage{color}
\usepackage{fancyvrb}
\newcommand{\VerbBar}{|}
\newcommand{\VERB}{\Verb[commandchars=\\\{\}]}
\DefineVerbatimEnvironment{Highlighting}{Verbatim}{commandchars=\\\{\}}
% Add ',fontsize=\small' for more characters per line
\usepackage{framed}
\definecolor{shadecolor}{RGB}{248,248,248}
\newenvironment{Shaded}{\begin{snugshade}}{\end{snugshade}}
\newcommand{\AlertTok}[1]{\textcolor[rgb]{0.94,0.16,0.16}{#1}}
\newcommand{\AnnotationTok}[1]{\textcolor[rgb]{0.56,0.35,0.01}{\textbf{\textit{#1}}}}
\newcommand{\AttributeTok}[1]{\textcolor[rgb]{0.77,0.63,0.00}{#1}}
\newcommand{\BaseNTok}[1]{\textcolor[rgb]{0.00,0.00,0.81}{#1}}
\newcommand{\BuiltInTok}[1]{#1}
\newcommand{\CharTok}[1]{\textcolor[rgb]{0.31,0.60,0.02}{#1}}
\newcommand{\CommentTok}[1]{\textcolor[rgb]{0.56,0.35,0.01}{\textit{#1}}}
\newcommand{\CommentVarTok}[1]{\textcolor[rgb]{0.56,0.35,0.01}{\textbf{\textit{#1}}}}
\newcommand{\ConstantTok}[1]{\textcolor[rgb]{0.00,0.00,0.00}{#1}}
\newcommand{\ControlFlowTok}[1]{\textcolor[rgb]{0.13,0.29,0.53}{\textbf{#1}}}
\newcommand{\DataTypeTok}[1]{\textcolor[rgb]{0.13,0.29,0.53}{#1}}
\newcommand{\DecValTok}[1]{\textcolor[rgb]{0.00,0.00,0.81}{#1}}
\newcommand{\DocumentationTok}[1]{\textcolor[rgb]{0.56,0.35,0.01}{\textbf{\textit{#1}}}}
\newcommand{\ErrorTok}[1]{\textcolor[rgb]{0.64,0.00,0.00}{\textbf{#1}}}
\newcommand{\ExtensionTok}[1]{#1}
\newcommand{\FloatTok}[1]{\textcolor[rgb]{0.00,0.00,0.81}{#1}}
\newcommand{\FunctionTok}[1]{\textcolor[rgb]{0.00,0.00,0.00}{#1}}
\newcommand{\ImportTok}[1]{#1}
\newcommand{\InformationTok}[1]{\textcolor[rgb]{0.56,0.35,0.01}{\textbf{\textit{#1}}}}
\newcommand{\KeywordTok}[1]{\textcolor[rgb]{0.13,0.29,0.53}{\textbf{#1}}}
\newcommand{\NormalTok}[1]{#1}
\newcommand{\OperatorTok}[1]{\textcolor[rgb]{0.81,0.36,0.00}{\textbf{#1}}}
\newcommand{\OtherTok}[1]{\textcolor[rgb]{0.56,0.35,0.01}{#1}}
\newcommand{\PreprocessorTok}[1]{\textcolor[rgb]{0.56,0.35,0.01}{\textit{#1}}}
\newcommand{\RegionMarkerTok}[1]{#1}
\newcommand{\SpecialCharTok}[1]{\textcolor[rgb]{0.00,0.00,0.00}{#1}}
\newcommand{\SpecialStringTok}[1]{\textcolor[rgb]{0.31,0.60,0.02}{#1}}
\newcommand{\StringTok}[1]{\textcolor[rgb]{0.31,0.60,0.02}{#1}}
\newcommand{\VariableTok}[1]{\textcolor[rgb]{0.00,0.00,0.00}{#1}}
\newcommand{\VerbatimStringTok}[1]{\textcolor[rgb]{0.31,0.60,0.02}{#1}}
\newcommand{\WarningTok}[1]{\textcolor[rgb]{0.56,0.35,0.01}{\textbf{\textit{#1}}}}

\usepackage{graphicx,grffile}
\makeatletter
\def\maxwidth{\ifdim\Gin@nat@width>\linewidth\linewidth\else\Gin@nat@width\fi}
\def\maxheight{\ifdim\Gin@nat@height>\textheight\textheight\else\Gin@nat@height\fi}
\makeatother
% Scale images if necessary, so that they will not overflow the page
% margins by default, and it is still possible to overwrite the defaults
% using explicit options in \includegraphics[width, height, ...]{}
\setkeys{Gin}{width=\maxwidth,height=\maxheight,keepaspectratio}


\title{Working from home: estimations without surveys \thanks{TBD}  }



\author{\Large Henning Holgersen\vspace{0.05in} \newline\normalsize\emph{Statistics Norway}  }


\date{}

\usepackage{titlesec}

\titleformat*{\section}{\normalsize\bfseries}
\titleformat*{\subsection}{\normalsize\itshape}
\titleformat*{\subsubsection}{\normalsize\itshape}
\titleformat*{\paragraph}{\normalsize\itshape}
\titleformat*{\subparagraph}{\normalsize\itshape}


\usepackage{natbib}
\bibliographystyle{apsr}
\usepackage[strings]{underscore} % protect underscores in most circumstances



\newtheorem{hypothesis}{Hypothesis}
\usepackage{setspace}


% set default figure placement to htbp
\makeatletter
\def\fps@figure{htbp}
\makeatother

\usepackage{hyperref}

% move the hyperref stuff down here, after header-includes, to allow for - \usepackage{hyperref}

\makeatletter
\@ifpackageloaded{hyperref}{}{%
\ifxetex
  \PassOptionsToPackage{hyphens}{url}\usepackage[setpagesize=false, % page size defined by xetex
              unicode=false, % unicode breaks when used with xetex
              xetex]{hyperref}
\else
  \PassOptionsToPackage{hyphens}{url}\usepackage[draft,unicode=true]{hyperref}
\fi
}

\@ifpackageloaded{color}{
    \PassOptionsToPackage{usenames,dvipsnames}{color}
}{%
    \usepackage[usenames,dvipsnames]{color}
}
\makeatother
\hypersetup{breaklinks=true,
            bookmarks=true,
            pdfauthor={Henning Holgersen (Statistics Norway)},
             pdfkeywords = {big data, surveys, work, text},  
            pdftitle={Working from home: estimations without surveys},
            colorlinks=true,
            citecolor=blue,
            urlcolor=blue,
            linkcolor=magenta,
            pdfborder={0 0 0}}
\urlstyle{same}  % don't use monospace font for urls

% Add an option for endnotes. -----


% add tightlist ----------
\providecommand{\tightlist}{%
\setlength{\itemsep}{0pt}\setlength{\parskip}{0pt}}

% add some other packages ----------

% \usepackage{multicol}
% This should regulate where figures float
% See: https://tex.stackexchange.com/questions/2275/keeping-tables-figures-close-to-where-they-are-mentioned
\usepackage[section]{placeins}


\begin{document}
	
% \pagenumbering{arabic}% resets `page` counter to 1 
%
% \maketitle

{% \usefont{T1}{pnc}{m}{n}
\setlength{\parindent}{0pt}
\thispagestyle{plain}
{\fontsize{18}{20}\selectfont\raggedright 
\maketitle  % title \par  

}

{
   \vskip 13.5pt\relax \normalsize\fontsize{11}{12} 
\textbf{\authorfont Henning Holgersen} \hskip 15pt \emph{\small Statistics Norway}   

}

}








\begin{abstract}

    \hbox{\vrule height .2pt width 39.14pc}

    \vskip 8.5pt % \small 

\noindent After the Covid-19 pandemic outbreak, working from home has become the
norm for many. Surveys have tried to capture the extent to which people
actually work from home, which is a type of problem surveys are good at.
We have attempted to use alternative approaches and data sources to say
something about who might be working from home. Alternative data sources
are rarely a drop-in replacement for surveys. Instead, they answer
different but related questions.


\vskip 8.5pt \noindent \emph{Keywords}: big data, surveys, work, text \par

    \hbox{\vrule height .2pt width 39.14pc}



\end{abstract}


\vskip -8.5pt


 % removetitleabstract

\noindent  

\hypertarget{introduction}{%
\section{Introduction}\label{introduction}}

The Covid-19 pandemic has led practically anyone who can to work from
home. A priori, we didn't have a lot of information about what
percentage of the workforce actually had this opportunity, and we knew
less about what characterizes these people. It is easy to guess that
office jobs generally can be done from home, but an office job is not
clearly defined in official statistics.

Surveys after the pandemic took effect have shown that more than 40
percent of the workforce now works from home. In contrast, some surveys
prior to the pandemic found that \textasciitilde{}30 percent of the
workforce were permittet to work from home at least on occasions.
Naturally, the two questions are different. During the pandemic,
practically all employers will let employees work from home if possible.
Before, working from home was somewhat of a perk for the employer to
grant if they had the opportunity and were so inclined.

In these situations, there are no obvious alternatives to surveys. Two
possibilities are evaluated here: Full text job ads, and annotating the
ISCO taxonomy by interpreting the documentation.

Job advertisements may mention in their ad if they have good options for
working from home. Working from home at times is a valuable perk for
many, and employers therefore may have an incentive to mention it. This
approach is contingent of job advertisements actually being available to
analyze, and it is very difficult to say anything about the total number
of remote-enabled jobs out there. Even saying something about the
relative frequency of remote jobs between industries, occupations etc.
requires calibration. But job advertisements are a rich source of data,
and if they are readily available, it would be extravagant to dismiss
them.

The ISCO taxonomy does not involve itself with wether or not the job can
be done remotely. The documentation, however, includes short but
informative descriptions of the job. Humans reading these descriptions
will often have an immediate understanding of the possibilities for
doing this job remotely. In order to classify each occupation as
remote-friendly or not. This is a fairly simple job to do, and the ISCO
documentation is available for anyone online. For anyone who already has
register or survey data which includes ISCO codes, this is a simple
approach to augment the data they already have.

The four data sources above (two surveys and to alternative sources),
answer four different but related questions:

\begin{itemize}
\tightlist
\item
  The surveys before covid asked employees if they had the
  \emph{opportunity} to work from home, and if so how often.
\item
  The surveys after covid primarily ask employees \emph{if} they worke
  from home (although some nuances exist)
\item
  The job ads can tell us about job openings the employer advertises as
  remote friendly
\item
  The ISCO annotations can tell if there is anything about the job
  itself that prevents it from being performed from home.
\end{itemize}

These differences can either be vital and not to be confused, relatively
minor nuances, or informative in themselves.

\hypertarget{what-the-surveys-say}{%
\section{What the surveys say}\label{what-the-surveys-say}}

\hypertarget{annotating-the-isco-standard}{%
\section{Annotating the ISCO
standard}\label{annotating-the-isco-standard}}

The ISCO standard organizes jobs into a set of groups according to the
tasks and duties undertaken in the job. Using the detailed task
descriptions listed in the ISCO-08 documentation, we try to provide a
assertion of whether an occupation is likely able to be performed from
home. To do this, we created a public labeling job through Amazon
Mechanical Turk \citep{Turk2020}. Each occupation was presented together
with a brief description. The exact question formulation was ``Can this
type of job likely be performed from a home office?'', and an example of
a job description could be:

\begin{quote}
Social work and counselling professionals provide advice and guidance to
individuals, families, groups, communities and organizations in response
to social and personal difficulties. They assist clients to develop
skills and access resources and support services needed to respond to
issues arising from unemployment, poverty, disability, addiction,
criminal and delinquent behaviour, marital and other problems.
\end{quote}

The respondent was asked to evaluate wether it was likely that the job
could be performed primarily from a private home. The alternatives were
``Yes'', ``No'' and ``Unknown'', which were provided with the following
description:

\begin{enumerate}
\def\labelenumi{\arabic{enumi}.}
\item
  \emph{Yes: This job can be performed primarily from an office in a
  private home}
\item
  \emph{No: Substatantial parts of this job must be performed outside
  the employees home}
\item
  \emph{Unknown: There is not enough information to decide}
\end{enumerate}

In order to reduce the serendipity in the labels, we acquired five
labels from different respondents for each occupation, and we provided
an \texttt{uncertain} option in addition to the \texttt{yes/no} options
in order to reduce arbitrary responses to uninformative occupation
descriptions. The final labels include an uncertainty measure which
shows that some of the occupations were evaluated differently by
different annotators, but no occupation was given a final label of
``Unknown'' which means we can treat the remote-friendly annotation as a
binary variable.

Since the job was on Mechanical Turk, there respondents were not subject
matter experts, and likely reside in different countries. This adds to
the importance of obtaining more than one label per occupation, but the
number of labels does not correct for possible cultural differences - it
is possible that some jobs that cannot be performed remotely in other
countries can be performed remotely in Norway. We should consider the
annotations as \texttt{international}, which is also true for the
ISCO-08 standard itself.

There are two additional approaches to the annotation approach: The ISCO
documentation includes examples of typical tasks, which can be
categorized on their own, and an aggregate measure of ``what fraction of
this job can be done at home'' can be calculated. Of course, this
measure will not be able to take into account the amount of time spent
at each task - there may be no substitute for surveys here.

The other approach, very similar, is to use ontologies like ESCO, which
includes tasks. This may open the possibility of calculating ``nearest
remote neighbor'', building on similarities between

\hypertarget{utilizing-job-ads}{%
\section{Utilizing job ads}\label{utilizing-job-ads}}

The market for job ads is relatively concentrated in Norway. For over 10
years, there has been two major actors in this market: the commercial
website finn.no, and the norwegian welfare administration, nav.no. Many
ads have been posted on both sites, but starting in 2018 the nav.no
website republishes ads from finn.no. In the summer of 2019, NAV
published an archive of job ads going back to 2002. This data set (one
giant csv file) contains the title and full text of the ad, information
about the employer including organizational number (ID from the central
business register), and an ISCO classification made by employees at NAV.

The job ads can tell us if the employer explicitly advertise rempte
possibilities, but we are faced with a question: How do we know? There
are basically three approaches.

Searching for specific words or phrases that are highly suggestive of
remote possibilities. This approach works if the language, like
norwegian, has a small number of words that indicate this. The solution
is low-tech, but may be surprisingly effective for certain languages.

The search can be amended by using word embeddings to find similar
words. Simple word embeddings, however, will not allow us to find
phrases that indicate remote possibilities. In order to use that
approach, advanced NLP techniques such as Named Entity Recognition must
be considered.

Since results are highly language dependent, we will limit our foray
into NLP.

Using a simple search, we find 3500 mentions of \emph{hjemmekontor} or
\emph{heimekontor}. This is very low considering there are 2.6 million
ads, but still enough to make some inferences.

\begin{Shaded}
\begin{Highlighting}[]
\CommentTok{#load('wfh.RData')}
\KeywordTok{loadd}\NormalTok{(wfh_plot_data)}

\NormalTok{wfh_plot_data }\OperatorTok\StringTok{ }
\StringTok{  }\KeywordTok{ggplot}\NormalTok{(}\KeywordTok{aes}\NormalTok{(aar, andel_stillinger)) }\OperatorTok{+}
\StringTok{  }\KeywordTok{geom_smooth}\NormalTok{()}
\end{Highlighting}
\end{Shaded}

\includegraphics{figs/unnamed-chunk-1.pdf}

That's all folks!

\begin{Shaded}
\begin{Highlighting}[]
\KeywordTok{head}\NormalTok{(iris)}
\end{Highlighting}
\end{Shaded}

\begin{verbatim}
##   Sepal.Length Sepal.Width Petal.Length Petal.Width Species
## 1          5.1         3.5          1.4         0.2  setosa
## 2          4.9         3.0          1.4         0.2  setosa
## 3          4.7         3.2          1.3         0.2  setosa
## 4          4.6         3.1          1.5         0.2  setosa
## 5          5.0         3.6          1.4         0.2  setosa
## 6          5.4         3.9          1.7         0.4  setosa
\end{verbatim}





\newpage
\singlespacing 
\end{document}
